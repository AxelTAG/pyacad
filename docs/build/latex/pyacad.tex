%% Generated by Sphinx.
\def\sphinxdocclass{report}
\documentclass[a4paper,10pt,english]{sphinxmanual}
\ifdefined\pdfpxdimen
   \let\sphinxpxdimen\pdfpxdimen\else\newdimen\sphinxpxdimen
\fi \sphinxpxdimen=.75bp\relax
\ifdefined\pdfimageresolution
    \pdfimageresolution= \numexpr \dimexpr1in\relax/\sphinxpxdimen\relax
\fi
%% let collapsible pdf bookmarks panel have high depth per default
\PassOptionsToPackage{bookmarksdepth=5}{hyperref}

\PassOptionsToPackage{booktabs}{sphinx}
\PassOptionsToPackage{colorrows}{sphinx}

\PassOptionsToPackage{warn}{textcomp}
\usepackage[utf8]{inputenc}
\ifdefined\DeclareUnicodeCharacter
% support both utf8 and utf8x syntaxes
  \ifdefined\DeclareUnicodeCharacterAsOptional
    \def\sphinxDUC#1{\DeclareUnicodeCharacter{"#1}}
  \else
    \let\sphinxDUC\DeclareUnicodeCharacter
  \fi
  \sphinxDUC{00A0}{\nobreakspace}
  \sphinxDUC{2500}{\sphinxunichar{2500}}
  \sphinxDUC{2502}{\sphinxunichar{2502}}
  \sphinxDUC{2514}{\sphinxunichar{2514}}
  \sphinxDUC{251C}{\sphinxunichar{251C}}
  \sphinxDUC{2572}{\textbackslash}
\fi
\usepackage{cmap}
\usepackage[T1]{fontenc}
\usepackage{amsmath,amssymb,amstext}
\usepackage{babel}



\usepackage{tgtermes}
\usepackage{tgheros}
\renewcommand{\ttdefault}{txtt}



\usepackage[Bjarne]{fncychap}
\usepackage{sphinx}

\fvset{fontsize=auto}
\usepackage{geometry}


% Include hyperref last.
\usepackage{hyperref}
% Fix anchor placement for figures with captions.
\usepackage{hypcap}% it must be loaded after hyperref.
% Set up styles of URL: it should be placed after hyperref.
\urlstyle{same}

\addto\captionsenglish{\renewcommand{\contentsname}{Contents:}}

\usepackage{sphinxmessages}
\setcounter{tocdepth}{1}



\title{pyacad}
\date{Aug 20, 2024}
\release{0.0.5}
\author{Kevin Axel Tagliaferri}
\newcommand{\sphinxlogo}{\vbox{}}
\renewcommand{\releasename}{Release}
\makeindex
\begin{document}

\ifdefined\shorthandoff
  \ifnum\catcode`\=\string=\active\shorthandoff{=}\fi
  \ifnum\catcode`\"=\active\shorthandoff{"}\fi
\fi

\pagestyle{empty}
\sphinxmaketitle
\pagestyle{plain}
\sphinxtableofcontents
\pagestyle{normal}
\phantomsection\label{\detokenize{index::doc}}


\sphinxAtStartPar
A package aimed to simplfy coding of Activex Automation Module of AutoCAD, based on pywin32 library.

\sphinxstepscope


\chapter{Getting started}
\label{\detokenize{gettingstarted:getting-started}}\label{\detokenize{gettingstarted::doc}}

\section{Installation}
\label{\detokenize{gettingstarted:installation}}
\sphinxAtStartPar
If you have pip:

\begin{sphinxVerbatim}[commandchars=\\\{\}]
\PYG{n}{pip} \PYG{n}{install} \PYG{n}{pyacad}
\end{sphinxVerbatim}


\section{Requirements}
\label{\detokenize{gettingstarted:requirements}}\begin{itemize}
\item {} 
\sphinxAtStartPar
pywin32

\end{itemize}


\section{Retrieving AutoCAD ActiveX documentation}
\label{\detokenize{gettingstarted:retrieving-autocad-activex-documentation}}
\sphinxAtStartPar
Visit the official Autodesk site:
\begin{itemize}
\item {} 
\sphinxAtStartPar
\sphinxhref{https://help.autodesk.com/view/OARX/2024/ENU/?guid=GUID-36BF58F3-537D-4B59-BEFE-2D0FEF5A4443}{AutoCAD 2024}

\item {} 
\sphinxAtStartPar
\sphinxhref{https://help.autodesk.com/view/OARX/2023/ENU/?guid=GUID-36BF58F3-537D-4B59-BEFE-2D0FEF5A4443}{AutoCAD 2023}

\end{itemize}

\sphinxstepscope


\chapter{Usage}
\label{\detokenize{usage:usage}}\label{\detokenize{usage::doc}}
\sphinxAtStartPar
For the following examples, we will use \sphinxcode{\sphinxupquote{Autocad}} and \sphinxcode{\sphinxupquote{APoint}}.

\begin{sphinxVerbatim}[commandchars=\\\{\}]
\PYG{k+kn}{from} \PYG{n+nn}{pyacad} \PYG{k+kn}{import} \PYG{n}{Autocad}
\end{sphinxVerbatim}


\section{Initializing the AutoCAD API:}
\label{\detokenize{usage:initializing-the-autocad-api}}
\begin{sphinxVerbatim}[commandchars=\\\{\}]
\PYG{n}{acad} \PYG{o}{=} \PYG{n}{Autocad}\PYG{p}{(}\PYG{p}{)}
\end{sphinxVerbatim}


\section{Retrieving active document name}
\label{\detokenize{usage:retrieving-active-document-name}}
\begin{sphinxVerbatim}[commandchars=\\\{\}]
\PYG{c+c1}{\PYGZsh{} Only the document name.}
\PYG{n}{acad}\PYG{o}{.}\PYG{n}{doc}\PYG{o}{.}\PYG{n}{Name}

\PYG{c+c1}{\PYGZsh{} For the document path.}
\PYG{n}{acad}\PYG{o}{.}\PYG{n}{doc}\PYG{o}{.}\PYG{n}{FullName}
\end{sphinxVerbatim}


\section{Adding objects to the active document}
\label{\detokenize{usage:adding-objects-to-the-active-document}}

\subsection{Points}
\label{\detokenize{usage:points}}
\begin{sphinxVerbatim}[commandchars=\\\{\}]
\PYG{n}{p0} \PYG{o}{=} \PYG{n}{APoint}\PYG{p}{(}\PYG{l+m+mi}{2}\PYG{p}{,} \PYG{l+m+mi}{1}\PYG{p}{)}
\PYG{n}{point} \PYG{o}{=} \PYG{n}{acad}\PYG{o}{.}\PYG{n}{model}\PYG{o}{.}\PYG{n}{AddPoint}\PYG{p}{(}\PYG{n}{p0}\PYG{p}{(}\PYG{p}{)}\PYG{p}{)}
\end{sphinxVerbatim}


\subsection{Lines}
\label{\detokenize{usage:lines}}
\begin{sphinxVerbatim}[commandchars=\\\{\}]
\PYG{n}{p0}\PYG{p}{,} \PYG{n}{p1} \PYG{o}{=} \PYG{n}{APoint}\PYG{p}{(}\PYG{l+m+mi}{1}\PYG{p}{,} \PYG{l+m+mi}{1}\PYG{p}{)}\PYG{p}{,} \PYG{n}{APoint}\PYG{p}{(}\PYG{l+m+mi}{2}\PYG{p}{,} \PYG{l+m+mi}{1}\PYG{p}{)}
\PYG{n}{line} \PYG{o}{=} \PYG{n}{acad}\PYG{o}{.}\PYG{n}{model}\PYG{o}{.}\PYG{n}{AddLine}\PYG{p}{(}\PYG{n}{p0}\PYG{p}{(}\PYG{p}{)}\PYG{p}{,} \PYG{n}{p1}\PYG{p}{(}\PYG{p}{)}\PYG{p}{)}
\end{sphinxVerbatim}


\subsection{Polylines}
\label{\detokenize{usage:polylines}}
\sphinxAtStartPar
For drawing polylines we need to use aDouble.

\begin{sphinxVerbatim}[commandchars=\\\{\}]
\PYG{k+kn}{from} \PYG{n+nn}{pyacad} \PYG{k+kn}{import} \PYG{n}{aDouble}
\PYG{n}{points} \PYG{o}{=} \PYG{n}{aDouble}\PYG{p}{(}\PYG{p}{[}\PYG{l+m+mi}{0}\PYG{p}{,} \PYG{l+m+mi}{0}\PYG{p}{,} \PYG{l+m+mi}{1}\PYG{p}{,} \PYG{l+m+mi}{0}\PYG{p}{,} \PYG{l+m+mi}{1}\PYG{p}{,} \PYG{l+m+mi}{1}\PYG{p}{,} \PYG{l+m+mi}{0}\PYG{p}{,} \PYG{l+m+mi}{1}\PYG{p}{,} \PYG{l+m+mi}{0}\PYG{p}{,} \PYG{l+m+mi}{0}\PYG{p}{]}\PYG{p}{)}
\PYG{n}{polyline} \PYG{o}{=} \PYG{n}{acad}\PYG{o}{.}\PYG{n}{model}\PYG{o}{.}\PYG{n}{AddLightweightPolyline}\PYG{p}{(}\PYG{n}{points}\PYG{p}{)}
\end{sphinxVerbatim}


\subsection{Circles}
\label{\detokenize{usage:circles}}
\begin{sphinxVerbatim}[commandchars=\\\{\}]
\PYG{n}{p0} \PYG{o}{=} \PYG{n}{APoint}\PYG{p}{(}\PYG{l+m+mi}{3}\PYG{p}{,} \PYG{l+m+mi}{1}\PYG{p}{)}
\PYG{n}{radius} \PYG{o}{=} \PYG{l+m+mf}{.5}
\PYG{n}{circle} \PYG{o}{=} \PYG{n}{acad}\PYG{o}{.}\PYG{n}{model}\PYG{o}{.}\PYG{n}{AddCircle}\PYG{p}{(}\PYG{n}{p0}\PYG{p}{(}\PYG{p}{)}\PYG{p}{,} \PYG{n}{radius}\PYG{p}{)}
\end{sphinxVerbatim}


\subsection{Text}
\label{\detokenize{usage:text}}
\begin{sphinxVerbatim}[commandchars=\\\{\}]
\PYG{n}{p0} \PYG{o}{=} \PYG{n}{APoint}\PYG{p}{(}\PYG{l+m+mi}{0}\PYG{p}{,} \PYG{l+m+mi}{3}\PYG{p}{)}
\PYG{n}{height} \PYG{o}{=} \PYG{l+m+mi}{1}
\PYG{n}{textstring} \PYG{o}{=} \PYG{l+s+s2}{\PYGZdq{}}\PYG{l+s+s2}{Hello World!}\PYG{l+s+s2}{\PYGZdq{}}
\PYG{n}{text} \PYG{o}{=} \PYG{n}{acad}\PYG{o}{.}\PYG{n}{model}\PYG{o}{.}\PYG{n}{AddText}\PYG{p}{(}\PYG{n}{textstring}\PYG{p}{,} \PYG{n}{p0}\PYG{p}{(}\PYG{p}{)}\PYG{p}{,} \PYG{n}{height}\PYG{p}{)}
\end{sphinxVerbatim}


\subsection{MultiLineText}
\label{\detokenize{usage:multilinetext}}
\begin{sphinxVerbatim}[commandchars=\\\{\}]
\PYG{n}{p0} \PYG{o}{=} \PYG{n}{APoint}\PYG{p}{(}\PYG{l+m+mi}{0}\PYG{p}{,} \PYG{l+m+mi}{4}\PYG{p}{)}
\PYG{n}{width} \PYG{o}{=} \PYG{l+m+mi}{1}
\PYG{n}{textstring} \PYG{o}{=} \PYG{l+s+s2}{\PYGZdq{}}\PYG{l+s+s2}{This is a MText.}\PYG{l+s+s2}{\PYGZdq{}}
\PYG{n}{mtext} \PYG{o}{=} \PYG{n}{acad}\PYG{o}{.}\PYG{n}{model}\PYG{o}{.}\PYG{n}{AddMText}\PYG{p}{(}\PYG{n}{p0}\PYG{p}{(}\PYG{p}{)}\PYG{p}{,} \PYG{n}{width}\PYG{p}{,} \PYG{n}{textstring}\PYG{p}{)}
\end{sphinxVerbatim}


\subsection{Hatch}
\label{\detokenize{usage:hatch}}
\sphinxAtStartPar
For drawing hatchs we need to use aDispatch.

\begin{sphinxVerbatim}[commandchars=\\\{\}]
\PYG{c+c1}{\PYGZsh{} Defining boundary.}
\PYG{n}{outer\PYGZus{}boundary} \PYG{o}{=} \PYG{p}{[}\PYG{p}{]}
\PYG{n}{outer\PYGZus{}boundary}\PYG{o}{.}\PYG{n}{append}\PYG{p}{(}\PYG{n}{acad}\PYG{o}{.}\PYG{n}{model}\PYG{o}{.}\PYG{n}{AddCircle}\PYG{p}{(}\PYG{n}{APoint}\PYG{p}{(}\PYG{l+m+mi}{0}\PYG{p}{,} \PYG{l+m+mi}{0}\PYG{p}{)}\PYG{p}{(}\PYG{p}{)}\PYG{p}{,} \PYG{l+m+mi}{1}\PYG{p}{)}\PYG{p}{)}
\PYG{n}{outer\PYGZus{}boundary\PYGZus{}dispatch} \PYG{o}{=} \PYG{n}{aDispatch}\PYG{p}{(}\PYG{n}{outer\PYGZus{}boundary}\PYG{p}{)}

\PYG{c+c1}{\PYGZsh{} Creating hatch and adding boundary.}
\PYG{n}{hatch} \PYG{o}{=} \PYG{n}{acad}\PYG{o}{.}\PYG{n}{model}\PYG{o}{.}\PYG{n}{AddHatch}\PYG{p}{(}\PYG{l+m+mi}{0}\PYG{p}{,} \PYG{l+s+s2}{\PYGZdq{}}\PYG{l+s+s2}{ANSI31}\PYG{l+s+s2}{\PYGZdq{}}\PYG{p}{,} \PYG{k+kc}{True}\PYG{p}{)}
\PYG{n}{hatch}\PYG{o}{.}\PYG{n}{AppendOuterLoop}\PYG{p}{(}\PYG{n}{outer\PYGZus{}boundary\PYGZus{}dispatch}\PYG{p}{)}
\PYG{n}{hatch}\PYG{o}{.}\PYG{n}{Evaluate}\PYG{p}{(}\PYG{p}{)}
\end{sphinxVerbatim}


\subsection{Aligned Dimension}
\label{\detokenize{usage:aligned-dimension}}
\begin{sphinxVerbatim}[commandchars=\\\{\}]
\PYG{n}{p0}\PYG{p}{,} \PYG{n}{p1}\PYG{p}{,} \PYG{n}{p2} \PYG{o}{=} \PYG{n}{APoint}\PYG{p}{(}\PYG{l+m+mi}{0}\PYG{p}{,} \PYG{l+m+mi}{4}\PYG{p}{)}\PYG{p}{,} \PYG{n}{APoint}\PYG{p}{(}\PYG{l+m+mi}{4}\PYG{p}{,} \PYG{l+m+mi}{4}\PYG{p}{)}\PYG{p}{,} \PYG{n}{APoint}\PYG{p}{(}\PYG{l+m+mi}{2}\PYG{p}{,} \PYG{l+m+mf}{4.5}\PYG{p}{)}
\PYG{n}{acad}\PYG{o}{.}\PYG{n}{model}\PYG{o}{.}\PYG{n}{AddDimAligned}\PYG{p}{(}\PYG{n}{p0}\PYG{p}{(}\PYG{p}{)}\PYG{p}{,} \PYG{n}{p1}\PYG{p}{(}\PYG{p}{)}\PYG{p}{,} \PYG{n}{p3}\PYG{p}{(}\PYG{p}{)}\PYG{p}{)}
\end{sphinxVerbatim}


\subsection{Further information about objects}
\label{\detokenize{usage:further-information-about-objects}}
\sphinxAtStartPar
For more information on creating objects, I recommend visiting the official Autodesk site on ActiveX Automation for your corresponding version of AutoCAD.
\begin{itemize}
\item {} 
\sphinxAtStartPar
\sphinxhref{https://help.autodesk.com/view/OARX/2024/ENU/?guid=GUID-36BF58F3-537D-4B59-BEFE-2D0FEF5A4443}{AutoCAD 2024}

\item {} 
\sphinxAtStartPar
\sphinxhref{https://help.autodesk.com/view/OARX/2023/ENU/?guid=GUID-36BF58F3-537D-4B59-BEFE-2D0FEF5A4443}{AutoCAD 2023}

\end{itemize}


\section{Retrieving over documents, layouts, layer, objects and more.}
\label{\detokenize{usage:retrieving-over-documents-layouts-layer-objects-and-more}}

\subsection{Retrieving documents from the AutoCAD API}
\label{\detokenize{usage:retrieving-documents-from-the-autocad-api}}
\begin{sphinxVerbatim}[commandchars=\\\{\}]
\PYG{n}{acad}\PYG{o}{.}\PYG{n}{iter\PYGZus{}documents}\PYG{p}{(}\PYG{p}{)}
\end{sphinxVerbatim}


\subsection{Retrieving blocks from the document}
\label{\detokenize{usage:retrieving-blocks-from-the-document}}
\begin{sphinxVerbatim}[commandchars=\\\{\}]
\PYG{n}{blocks} \PYG{o}{=} \PYG{n}{acad}\PYG{o}{.}\PYG{n}{iter\PYGZus{}blocks}\PYG{p}{(}\PYG{p}{)}
\end{sphinxVerbatim}

\sphinxAtStartPar
Alternatively, you can pass a specific document using the document parameter, which should be of the type returned by the app.iter\_documents() function.

\begin{sphinxVerbatim}[commandchars=\\\{\}]
\PYG{n}{docs} \PYG{o}{=} \PYG{p}{[}\PYG{o}{*}\PYG{n}{acad}\PYG{o}{.}\PYG{n}{iter\PYGZus{}documents}\PYG{p}{(}\PYG{p}{)}\PYG{p}{]}
\PYG{n}{doc\PYGZus{}selected} \PYG{o}{=} \PYG{n}{docs}\PYG{p}{[}\PYG{l+m+mi}{0}\PYG{p}{]}  \PYG{c+c1}{\PYGZsh{} 0 if you want select first document of list.}
\PYG{n}{blocks} \PYG{o}{=} \PYG{n}{acad}\PYG{o}{.}\PYG{n}{iter\PYGZus{}blocks}\PYG{p}{(}\PYG{n}{document}\PYG{o}{=}\PYG{n}{doc\PYGZus{}selected}\PYG{p}{)}
\end{sphinxVerbatim}


\subsection{Retrieving dimension styles from the document}
\label{\detokenize{usage:retrieving-dimension-styles-from-the-document}}
\begin{sphinxVerbatim}[commandchars=\\\{\}]
\PYG{n}{dim\PYGZus{}styles} \PYG{o}{=} \PYG{n}{acad}\PYG{o}{.}\PYG{n}{iter\PYGZus{}dim\PYGZus{}styles}\PYG{p}{(}\PYG{p}{)}
\end{sphinxVerbatim}

\sphinxAtStartPar
You can also do it in the same way as shown in iter\_blocks().


\subsection{Retrieving layers from the document}
\label{\detokenize{usage:retrieving-layers-from-the-document}}
\begin{sphinxVerbatim}[commandchars=\\\{\}]
\PYG{n}{layers} \PYG{o}{=} \PYG{n}{acad}\PYG{o}{.}\PYG{n}{iter\PYGZus{}layers}\PYG{p}{(}\PYG{p}{)}
\end{sphinxVerbatim}

\sphinxAtStartPar
You can also do it in the same way as shown in iter\_blocks().


\subsection{Retrieving layouts from the document}
\label{\detokenize{usage:retrieving-layouts-from-the-document}}
\begin{sphinxVerbatim}[commandchars=\\\{\}]
\PYG{n}{layouts} \PYG{o}{=} \PYG{n}{acad}\PYG{o}{.}\PYG{n}{iter\PYGZus{}layouts}\PYG{p}{(}\PYG{p}{)}
\end{sphinxVerbatim}

\sphinxAtStartPar
You can also do it in the same way as shown in iter\_blocks().


\subsection{Retrieving objects from the document}
\label{\detokenize{usage:retrieving-objects-from-the-document}}
\sphinxAtStartPar
You can iterate over the objects in a drawing.

\begin{sphinxVerbatim}[commandchars=\\\{\}]
\PYG{n}{objects} \PYG{o}{=} \PYG{n}{acad}\PYG{o}{.}\PYG{n}{iter\PYGZus{}objects}\PYG{p}{(}\PYG{p}{)}
\end{sphinxVerbatim}

\sphinxAtStartPar
Also you can filter for a concrete obejct type.

\begin{sphinxVerbatim}[commandchars=\\\{\}]
\PYG{n}{text\PYGZus{}obejects} \PYG{o}{=} \PYG{n}{acad}\PYG{o}{.}\PYG{n}{iter\PYGZus{}objects}\PYG{p}{(}\PYG{l+s+s2}{\PYGZdq{}}\PYG{l+s+s2}{Text}\PYG{l+s+s2}{\PYGZdq{}}\PYG{p}{)}
\end{sphinxVerbatim}


\subsection{Retrieving text styles from the document}
\label{\detokenize{usage:retrieving-text-styles-from-the-document}}
\begin{sphinxVerbatim}[commandchars=\\\{\}]
\PYG{n}{text\PYGZus{}styles} \PYG{o}{=} \PYG{n}{acad}\PYG{o}{.}\PYG{n}{iter\PYGZus{}text\PYGZus{}styles}\PYG{p}{(}\PYG{p}{)}
\end{sphinxVerbatim}

\sphinxAtStartPar
You can also do it in the same way as shown in iter\_blocks().

\sphinxstepscope


\chapter{References}
\label{\detokenize{API:references}}\label{\detokenize{API::doc}}
\sphinxAtStartPar
Since Read the Docs compiles documentation in a Linux environment, the automatic generation of documentation using
Sphinx is hindered because pywin32 only supports Win32 environments. Given this, we have compiled the documentation
locally and uploaded it in PDF format to the following link at GitHub.

\sphinxAtStartPar
\sphinxhref{https://github.com/AxelTAG/pyacad/tree/main/docs/build/latex/pyacad.pdf}{API Documentation}
\index{module@\spxentry{module}!pyacad.Autocad@\spxentry{pyacad.Autocad}}\index{pyacad.Autocad@\spxentry{pyacad.Autocad}!module@\spxentry{module}}\index{Autocad (class in pyacad.Autocad)@\spxentry{Autocad}\spxextra{class in pyacad.Autocad}}\phantomsection\label{\detokenize{API:module-pyacad.Autocad}}

\begin{fulllineitems}
\phantomsection\label{\detokenize{API:pyacad.Autocad.Autocad}}
\pysigstartsignatures
\pysiglinewithargsret{\sphinxbfcode{\sphinxupquote{class\DUrole{w}{ }}}\sphinxcode{\sphinxupquote{pyacad.Autocad.}}\sphinxbfcode{\sphinxupquote{Autocad}}}{\sphinxparam{\DUrole{n}{create\_if\_not\_exist}\DUrole{p}{:}\DUrole{w}{ }\DUrole{n}{bool}\DUrole{w}{ }\DUrole{o}{=}\DUrole{w}{ }\DUrole{default_value}{True}}\sphinxparamcomma \sphinxparam{\DUrole{n}{visible}\DUrole{p}{:}\DUrole{w}{ }\DUrole{n}{bool}\DUrole{w}{ }\DUrole{o}{=}\DUrole{w}{ }\DUrole{default_value}{True}}}{}
\pysigstopsignatures
\sphinxAtStartPar
Main class of AutoCAD app.
\index{app (pyacad.Autocad.Autocad property)@\spxentry{app}\spxextra{pyacad.Autocad.Autocad property}}

\begin{fulllineitems}
\phantomsection\label{\detokenize{API:pyacad.Autocad.Autocad.app}}
\pysigstartsignatures
\pysigline{\sphinxbfcode{\sphinxupquote{property\DUrole{w}{ }}}\sphinxbfcode{\sphinxupquote{app}}}
\pysigstopsignatures
\sphinxAtStartPar
Creates/gets and returns AutoCAD Application.

\end{fulllineitems}

\index{doc (pyacad.Autocad.Autocad property)@\spxentry{doc}\spxextra{pyacad.Autocad.Autocad property}}

\begin{fulllineitems}
\phantomsection\label{\detokenize{API:pyacad.Autocad.Autocad.doc}}
\pysigstartsignatures
\pysigline{\sphinxbfcode{\sphinxupquote{property\DUrole{w}{ }}}\sphinxbfcode{\sphinxupquote{doc}}}
\pysigstopsignatures
\sphinxAtStartPar
Returns active document.

\end{fulllineitems}

\index{iter\_blocks() (pyacad.Autocad.Autocad method)@\spxentry{iter\_blocks()}\spxextra{pyacad.Autocad.Autocad method}}

\begin{fulllineitems}
\phantomsection\label{\detokenize{API:pyacad.Autocad.Autocad.iter_blocks}}
\pysigstartsignatures
\pysiglinewithargsret{\sphinxbfcode{\sphinxupquote{iter\_blocks}}}{\sphinxparam{\DUrole{n}{document}\DUrole{o}{=}\DUrole{default_value}{None}}}{}
\pysigstopsignatures
\sphinxAtStartPar
Iterate over the existing block definitions in the specified document.
\begin{quote}\begin{description}
\sphinxlineitem{Parameters}
\sphinxAtStartPar
\sphinxstyleliteralstrong{\sphinxupquote{document}} (\sphinxstyleliteralemphasis{\sphinxupquote{win32com.client.CDispatch}}\sphinxstyleliteralemphasis{\sphinxupquote{, }}\sphinxstyleliteralemphasis{\sphinxupquote{None}}) \textendash{} COM object returned from the AutoCAD application. If \sphinxtitleref{None} is specified,
the active document is used.

\sphinxlineitem{Yield}
\sphinxAtStartPar
Generator of block definitions in the specified document.

\sphinxlineitem{Return type}
\sphinxAtStartPar
generator

\end{description}\end{quote}

\end{fulllineitems}

\index{iter\_dim\_styles() (pyacad.Autocad.Autocad method)@\spxentry{iter\_dim\_styles()}\spxextra{pyacad.Autocad.Autocad method}}

\begin{fulllineitems}
\phantomsection\label{\detokenize{API:pyacad.Autocad.Autocad.iter_dim_styles}}
\pysigstartsignatures
\pysiglinewithargsret{\sphinxbfcode{\sphinxupquote{iter\_dim\_styles}}}{\sphinxparam{\DUrole{n}{document}\DUrole{o}{=}\DUrole{default_value}{None}}}{}
\pysigstopsignatures
\sphinxAtStartPar
Iterate over the existing dimension styles in the specified document.
\begin{quote}\begin{description}
\sphinxlineitem{Parameters}
\sphinxAtStartPar
\sphinxstyleliteralstrong{\sphinxupquote{document}} (\sphinxstyleliteralemphasis{\sphinxupquote{win32com.client.CDispatch}}\sphinxstyleliteralemphasis{\sphinxupquote{, }}\sphinxstyleliteralemphasis{\sphinxupquote{None}}) \textendash{} COM object returned from the AutoCAD application. If \sphinxtitleref{None} is specified,
the active document is used.

\sphinxlineitem{Yield}
\sphinxAtStartPar
Generator of dimension styles in the specified document.

\sphinxlineitem{Return type}
\sphinxAtStartPar
generator

\end{description}\end{quote}

\end{fulllineitems}

\index{iter\_layers() (pyacad.Autocad.Autocad method)@\spxentry{iter\_layers()}\spxextra{pyacad.Autocad.Autocad method}}

\begin{fulllineitems}
\phantomsection\label{\detokenize{API:pyacad.Autocad.Autocad.iter_layers}}
\pysigstartsignatures
\pysiglinewithargsret{\sphinxbfcode{\sphinxupquote{iter\_layers}}}{\sphinxparam{\DUrole{n}{document}\DUrole{o}{=}\DUrole{default_value}{None}}}{}
\pysigstopsignatures
\sphinxAtStartPar
Iterate over the layers in the specified document.
\begin{quote}\begin{description}
\sphinxlineitem{Parameters}
\sphinxAtStartPar
\sphinxstyleliteralstrong{\sphinxupquote{document}} (\sphinxstyleliteralemphasis{\sphinxupquote{win32com.client.CDispatch}}\sphinxstyleliteralemphasis{\sphinxupquote{, }}\sphinxstyleliteralemphasis{\sphinxupquote{None}}) \textendash{} COM object returned from the AutoCAD application. If \sphinxtitleref{None} is specified,
the active document is used.

\sphinxlineitem{Yield}
\sphinxAtStartPar
Generator of layers in the specified document.

\sphinxlineitem{Return type}
\sphinxAtStartPar
generator

\end{description}\end{quote}

\end{fulllineitems}

\index{iter\_layouts() (pyacad.Autocad.Autocad method)@\spxentry{iter\_layouts()}\spxextra{pyacad.Autocad.Autocad method}}

\begin{fulllineitems}
\phantomsection\label{\detokenize{API:pyacad.Autocad.Autocad.iter_layouts}}
\pysigstartsignatures
\pysiglinewithargsret{\sphinxbfcode{\sphinxupquote{iter\_layouts}}}{\sphinxparam{\DUrole{n}{document}\DUrole{o}{=}\DUrole{default_value}{None}}\sphinxparamcomma \sphinxparam{\DUrole{n}{skip\_model}\DUrole{o}{=}\DUrole{default_value}{False}}}{}
\pysigstopsignatures
\sphinxAtStartPar
Iterate over the layouts in the specified document.
\begin{quote}\begin{description}
\sphinxlineitem{Parameters}\begin{itemize}
\item {} 
\sphinxAtStartPar
\sphinxstyleliteralstrong{\sphinxupquote{document}} (\sphinxstyleliteralemphasis{\sphinxupquote{win32com.client.CDispatch}}\sphinxstyleliteralemphasis{\sphinxupquote{, }}\sphinxstyleliteralemphasis{\sphinxupquote{None}}) \textendash{} COM object returned from the AutoCAD application. If \sphinxtitleref{None} is specified,
the active document is used.

\item {} 
\sphinxAtStartPar
\sphinxstyleliteralstrong{\sphinxupquote{skip\_model}} (\sphinxstyleliteralemphasis{\sphinxupquote{bool}}) \textendash{} Whether to skip the model layout. Defaults to \sphinxtitleref{False}.

\end{itemize}

\sphinxlineitem{Yield}
\sphinxAtStartPar
Generator of layouts in the specified document.

\sphinxlineitem{Return type}
\sphinxAtStartPar
generator

\end{description}\end{quote}

\end{fulllineitems}

\index{iter\_objects() (pyacad.Autocad.Autocad method)@\spxentry{iter\_objects()}\spxextra{pyacad.Autocad.Autocad method}}

\begin{fulllineitems}
\phantomsection\label{\detokenize{API:pyacad.Autocad.Autocad.iter_objects}}
\pysigstartsignatures
\pysiglinewithargsret{\sphinxbfcode{\sphinxupquote{iter\_objects}}}{\sphinxparam{\DUrole{n}{block}\DUrole{o}{=}\DUrole{default_value}{None}}\sphinxparamcomma \sphinxparam{\DUrole{n}{filter\_for}\DUrole{o}{=}\DUrole{default_value}{None}}\sphinxparamcomma \sphinxparam{\DUrole{n}{limit}\DUrole{o}{=}\DUrole{default_value}{None}}}{}
\pysigstopsignatures
\sphinxAtStartPar
Iterate over the objects in a specified ‘block’.
\begin{quote}\begin{description}
\sphinxlineitem{Parameters}\begin{itemize}
\item {} 
\sphinxAtStartPar
\sphinxstyleliteralstrong{\sphinxupquote{block}} (\sphinxstyleliteralemphasis{\sphinxupquote{win32com.client.CDispatch}}\sphinxstyleliteralemphasis{\sphinxupquote{, }}\sphinxstyleliteralemphasis{\sphinxupquote{None}}) \textendash{} COM object returned from the AutoCAD application. If \sphinxtitleref{None} is specified,
the active layout is used.

\item {} 
\sphinxAtStartPar
\sphinxstyleliteralstrong{\sphinxupquote{filter\_for}} (\sphinxstyleliteralemphasis{\sphinxupquote{list}}\sphinxstyleliteralemphasis{\sphinxupquote{ of }}\sphinxstyleliteralemphasis{\sphinxupquote{str}}\sphinxstyleliteralemphasis{\sphinxupquote{, }}\sphinxstyleliteralemphasis{\sphinxupquote{tuple}}\sphinxstyleliteralemphasis{\sphinxupquote{ of }}\sphinxstyleliteralemphasis{\sphinxupquote{str}}\sphinxstyleliteralemphasis{\sphinxupquote{, }}\sphinxstyleliteralemphasis{\sphinxupquote{None}}) \textendash{} A filter for specific object types. Can be a list or tuple of strings.
If \sphinxtitleref{None}, no filtering is applied.

\item {} 
\sphinxAtStartPar
\sphinxstyleliteralstrong{\sphinxupquote{limit}} (\sphinxstyleliteralemphasis{\sphinxupquote{int}}\sphinxstyleliteralemphasis{\sphinxupquote{, }}\sphinxstyleliteralemphasis{\sphinxupquote{None}}) \textendash{} The maximum number of objects to iterate over. If \sphinxtitleref{None}, no limit is applied.

\end{itemize}

\sphinxlineitem{Yield}
\sphinxAtStartPar
Generator of objects in the specified layout or active layout if none is specified.

\sphinxlineitem{Return type}
\sphinxAtStartPar
generator

\end{description}\end{quote}

\end{fulllineitems}

\index{iter\_text\_styles() (pyacad.Autocad.Autocad method)@\spxentry{iter\_text\_styles()}\spxextra{pyacad.Autocad.Autocad method}}

\begin{fulllineitems}
\phantomsection\label{\detokenize{API:pyacad.Autocad.Autocad.iter_text_styles}}
\pysigstartsignatures
\pysiglinewithargsret{\sphinxbfcode{\sphinxupquote{iter\_text\_styles}}}{\sphinxparam{\DUrole{n}{document}\DUrole{o}{=}\DUrole{default_value}{None}}}{}
\pysigstopsignatures
\sphinxAtStartPar
Iterate over the existing text styles in the specified document.
\begin{quote}\begin{description}
\sphinxlineitem{Parameters}
\sphinxAtStartPar
\sphinxstyleliteralstrong{\sphinxupquote{document}} (\sphinxstyleliteralemphasis{\sphinxupquote{win32com.client.CDispatch}}\sphinxstyleliteralemphasis{\sphinxupquote{, }}\sphinxstyleliteralemphasis{\sphinxupquote{None}}) \textendash{} COM object returned from the AutoCAD application. If \sphinxtitleref{None} is specified,
the active document is used.

\sphinxlineitem{Yield}
\sphinxAtStartPar
Generator of text styles in the specified document.

\sphinxlineitem{Return type}
\sphinxAtStartPar
generator

\end{description}\end{quote}

\end{fulllineitems}

\index{model (pyacad.Autocad.Autocad property)@\spxentry{model}\spxextra{pyacad.Autocad.Autocad property}}

\begin{fulllineitems}
\phantomsection\label{\detokenize{API:pyacad.Autocad.Autocad.model}}
\pysigstartsignatures
\pysigline{\sphinxbfcode{\sphinxupquote{property\DUrole{w}{ }}}\sphinxbfcode{\sphinxupquote{model}}}
\pysigstopsignatures
\sphinxAtStartPar
Returns active model space of curring document.

\end{fulllineitems}


\end{fulllineitems}



\bigskip\hrule\bigskip

\index{module@\spxentry{module}!pyacad.APoint@\spxentry{pyacad.APoint}}\index{pyacad.APoint@\spxentry{pyacad.APoint}!module@\spxentry{module}}\index{APoint (class in pyacad.APoint)@\spxentry{APoint}\spxextra{class in pyacad.APoint}}\phantomsection\label{\detokenize{API:module-pyacad.APoint}}

\begin{fulllineitems}
\phantomsection\label{\detokenize{API:pyacad.APoint.APoint}}
\pysigstartsignatures
\pysiglinewithargsret{\sphinxbfcode{\sphinxupquote{class\DUrole{w}{ }}}\sphinxcode{\sphinxupquote{pyacad.APoint.}}\sphinxbfcode{\sphinxupquote{APoint}}}{\sphinxparam{\DUrole{n}{x}\DUrole{p}{:}\DUrole{w}{ }\DUrole{n}{float}}\sphinxparamcomma \sphinxparam{\DUrole{n}{y}\DUrole{p}{:}\DUrole{w}{ }\DUrole{n}{float}\DUrole{w}{ }\DUrole{o}{=}\DUrole{w}{ }\DUrole{default_value}{0}}\sphinxparamcomma \sphinxparam{\DUrole{n}{z}\DUrole{p}{:}\DUrole{w}{ }\DUrole{n}{float}\DUrole{w}{ }\DUrole{o}{=}\DUrole{w}{ }\DUrole{default_value}{0}}}{}
\pysigstopsignatures
\sphinxAtStartPar
3D point with basic geometric operations and support for passing as a
parameter for \sphinxtitleref{AutoCAD} Automation functions.
\begin{quote}\begin{description}
\sphinxlineitem{Variables}\begin{itemize}
\item {} 
\sphinxAtStartPar
\sphinxstyleliteralstrong{\sphinxupquote{x}} (\sphinxstyleliteralemphasis{\sphinxupquote{int}}\sphinxstyleliteralemphasis{\sphinxupquote{, }}\sphinxstyleliteralemphasis{\sphinxupquote{float}}\sphinxstyleliteralemphasis{\sphinxupquote{, }}\sphinxstyleliteralemphasis{\sphinxupquote{list}}\sphinxstyleliteralemphasis{\sphinxupquote{ of }}\sphinxstyleliteralemphasis{\sphinxupquote{int}}\sphinxstyleliteralemphasis{\sphinxupquote{, }}\sphinxstyleliteralemphasis{\sphinxupquote{list}}\sphinxstyleliteralemphasis{\sphinxupquote{ of }}\sphinxstyleliteralemphasis{\sphinxupquote{float}}\sphinxstyleliteralemphasis{\sphinxupquote{, }}\sphinxstyleliteralemphasis{\sphinxupquote{tuple}}\sphinxstyleliteralemphasis{\sphinxupquote{ of }}\sphinxstyleliteralemphasis{\sphinxupquote{int}}\sphinxstyleliteralemphasis{\sphinxupquote{, }}\sphinxstyleliteralemphasis{\sphinxupquote{tuple}}\sphinxstyleliteralemphasis{\sphinxupquote{ of }}\sphinxstyleliteralemphasis{\sphinxupquote{float}}) \textendash{} The X coordinate of the point.

\item {} 
\sphinxAtStartPar
\sphinxstyleliteralstrong{\sphinxupquote{y}} (\sphinxstyleliteralemphasis{\sphinxupquote{int}}\sphinxstyleliteralemphasis{\sphinxupquote{, }}\sphinxstyleliteralemphasis{\sphinxupquote{float}}\sphinxstyleliteralemphasis{\sphinxupquote{, }}\sphinxstyleliteralemphasis{\sphinxupquote{None}}) \textendash{} The Y coordinate of the point.

\item {} 
\sphinxAtStartPar
\sphinxstyleliteralstrong{\sphinxupquote{z}} (\sphinxstyleliteralemphasis{\sphinxupquote{int}}\sphinxstyleliteralemphasis{\sphinxupquote{, }}\sphinxstyleliteralemphasis{\sphinxupquote{float}}\sphinxstyleliteralemphasis{\sphinxupquote{, }}\sphinxstyleliteralemphasis{\sphinxupquote{None}}) \textendash{} The Z coordinate of the point.

\end{itemize}

\end{description}\end{quote}
\index{distance\_to() (pyacad.APoint.APoint method)@\spxentry{distance\_to()}\spxextra{pyacad.APoint.APoint method}}

\begin{fulllineitems}
\phantomsection\label{\detokenize{API:pyacad.APoint.APoint.distance_to}}
\pysigstartsignatures
\pysiglinewithargsret{\sphinxbfcode{\sphinxupquote{distance\_to}}}{\sphinxparam{\DUrole{n}{other}}}{}
\pysigstopsignatures
\sphinxAtStartPar
Calculate the distance to another 3D point.
\begin{quote}\begin{description}
\sphinxlineitem{Parameters}
\sphinxAtStartPar
\sphinxstyleliteralstrong{\sphinxupquote{other}} (\sphinxstyleliteralemphasis{\sphinxupquote{Point3D}}) \textendash{} The other Point3D object to calculate the distance to.

\sphinxlineitem{Returns}
\sphinxAtStartPar
The distance between the two points.

\sphinxlineitem{Return type}
\sphinxAtStartPar
float

\end{description}\end{quote}

\end{fulllineitems}


\end{fulllineitems}

\index{distance() (in module pyacad.APoint)@\spxentry{distance()}\spxextra{in module pyacad.APoint}}

\begin{fulllineitems}
\phantomsection\label{\detokenize{API:pyacad.APoint.distance}}
\pysigstartsignatures
\pysiglinewithargsret{\sphinxcode{\sphinxupquote{pyacad.APoint.}}\sphinxbfcode{\sphinxupquote{distance}}}{\sphinxparam{\DUrole{n}{p1}}\sphinxparamcomma \sphinxparam{\DUrole{n}{p2}}}{}
\pysigstopsignatures
\sphinxAtStartPar
Calculate the distance to another 3D point.
\begin{quote}\begin{description}
\sphinxlineitem{Parameters}\begin{itemize}
\item {} 
\sphinxAtStartPar
\sphinxstyleliteralstrong{\sphinxupquote{p1}} (\sphinxstyleliteralemphasis{\sphinxupquote{Point3D}}) \textendash{} The other Point3D object to calculate the distance to.

\item {} 
\sphinxAtStartPar
\sphinxstyleliteralstrong{\sphinxupquote{p2}} (\sphinxstyleliteralemphasis{\sphinxupquote{Point3D}}) \textendash{} The other Point3D object to calculate the distance to.

\end{itemize}

\sphinxlineitem{Returns}
\sphinxAtStartPar
The distance between the two points.

\sphinxlineitem{Return type}
\sphinxAtStartPar
float

\end{description}\end{quote}

\end{fulllineitems}



\bigskip\hrule\bigskip

\index{module@\spxentry{module}!pyacad.Types@\spxentry{pyacad.Types}}\index{pyacad.Types@\spxentry{pyacad.Types}!module@\spxentry{module}}\index{aDispatch() (in module pyacad.Types)@\spxentry{aDispatch()}\spxextra{in module pyacad.Types}}\phantomsection\label{\detokenize{API:module-pyacad.Types}}

\begin{fulllineitems}
\phantomsection\label{\detokenize{API:pyacad.Types.aDispatch}}
\pysigstartsignatures
\pysiglinewithargsret{\sphinxcode{\sphinxupquote{pyacad.Types.}}\sphinxbfcode{\sphinxupquote{aDispatch}}}{\sphinxparam{\DUrole{n}{object}}}{}
\pysigstopsignatures
\sphinxAtStartPar
Packs a win32 object into Variant Array.
\begin{quote}\begin{description}
\sphinxlineitem{Parameters}
\sphinxAtStartPar
\sphinxstyleliteralstrong{\sphinxupquote{object}} \textendash{} Win32 object.

\sphinxlineitem{Returns}
\sphinxAtStartPar
An array variant suitable for AutoCAD.

\sphinxlineitem{Return type}
\sphinxAtStartPar
VARIANT

\end{description}\end{quote}

\end{fulllineitems}

\index{aDouble() (in module pyacad.Types)@\spxentry{aDouble()}\spxextra{in module pyacad.Types}}

\begin{fulllineitems}
\phantomsection\label{\detokenize{API:pyacad.Types.aDouble}}
\pysigstartsignatures
\pysiglinewithargsret{\sphinxcode{\sphinxupquote{pyacad.Types.}}\sphinxbfcode{\sphinxupquote{aDouble}}}{\sphinxparam{\DUrole{n}{xyz}}}{}
\pysigstopsignatures
\sphinxAtStartPar
Packs a list or tuple of floats into an array for passing to AutoCAD.
\begin{quote}\begin{description}
\sphinxlineitem{Parameters}
\sphinxAtStartPar
\sphinxstyleliteralstrong{\sphinxupquote{xyz}} (\sphinxstyleliteralemphasis{\sphinxupquote{list}}\sphinxstyleliteralemphasis{\sphinxupquote{ of }}\sphinxstyleliteralemphasis{\sphinxupquote{float}}\sphinxstyleliteralemphasis{\sphinxupquote{, }}\sphinxstyleliteralemphasis{\sphinxupquote{tuple}}\sphinxstyleliteralemphasis{\sphinxupquote{ of }}\sphinxstyleliteralemphasis{\sphinxupquote{float}}) \textendash{} A tuple or list of floats to be packed into an array.

\sphinxlineitem{Returns}
\sphinxAtStartPar
An array variant suitable for AutoCAD.

\sphinxlineitem{Return type}
\sphinxAtStartPar
VARIANT

\end{description}\end{quote}

\end{fulllineitems}

\index{aInt() (in module pyacad.Types)@\spxentry{aInt()}\spxextra{in module pyacad.Types}}

\begin{fulllineitems}
\phantomsection\label{\detokenize{API:pyacad.Types.aInt}}
\pysigstartsignatures
\pysiglinewithargsret{\sphinxcode{\sphinxupquote{pyacad.Types.}}\sphinxbfcode{\sphinxupquote{aInt}}}{\sphinxparam{\DUrole{n}{xyz}}}{}
\pysigstopsignatures
\sphinxAtStartPar
Packs list of floats into an array for passing to AutoCAD (same as aDouble).
\begin{quote}\begin{description}
\sphinxlineitem{Parameters}
\sphinxAtStartPar
\sphinxstyleliteralstrong{\sphinxupquote{xyz}} (\sphinxstyleliteralemphasis{\sphinxupquote{list}}\sphinxstyleliteralemphasis{\sphinxupquote{ of }}\sphinxstyleliteralemphasis{\sphinxupquote{float}}\sphinxstyleliteralemphasis{\sphinxupquote{, }}\sphinxstyleliteralemphasis{\sphinxupquote{tuple}}\sphinxstyleliteralemphasis{\sphinxupquote{ of }}\sphinxstyleliteralemphasis{\sphinxupquote{float}}) \textendash{} List of floats or integers.

\sphinxlineitem{Returns}
\sphinxAtStartPar
An array variant suitable for AutoCAD.

\sphinxlineitem{Return type}
\sphinxAtStartPar
VARIANT

\end{description}\end{quote}

\end{fulllineitems}


\sphinxAtStartPar
٩(˘◡˘)۶ If you want to pay me for a beer, coffee, or something else: \sphinxhref{https://www.paypal.com/paypalme/KevinAxelTagliaferri}{C|\_|}


\renewcommand{\indexname}{Python Module Index}
\begin{sphinxtheindex}
\let\bigletter\sphinxstyleindexlettergroup
\bigletter{p}
\item\relax\sphinxstyleindexentry{pyacad.APoint}\sphinxstyleindexpageref{API:\detokenize{module-pyacad.APoint}}
\item\relax\sphinxstyleindexentry{pyacad.Autocad}\sphinxstyleindexpageref{API:\detokenize{module-pyacad.Autocad}}
\item\relax\sphinxstyleindexentry{pyacad.Types}\sphinxstyleindexpageref{API:\detokenize{module-pyacad.Types}}
\end{sphinxtheindex}

\renewcommand{\indexname}{Index}
\printindex
\end{document}